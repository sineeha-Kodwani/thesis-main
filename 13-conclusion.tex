\chapter{Conclusion}\label{chap:LaTeXAdvice}

This thesis presents a comprehensive formalization of AVL trees within the HOL4 theorem prover, contributing to the broader effort of enriching HOL4's library of verified data structures. The formalization covers key aspects of AVL trees, including recursive data structures, balance conditions, height constraints, and the associated balancing operations. Through this work, we have formalized the core properties that ensure AVL trees maintain their logarithmic height and efficiency for search, insertion, and deletion operations.

A central achievement of this thesis is the detailed specification and proof of AVL tree correctness, particularly focusing on the invariant that ensures the height difference between subtrees remains bounded. By formalizing the recursive relationships between nodes, this work reinforces the importance of AVL trees in guaranteeing optimal performance for balanced binary search trees.

In comparing the HOL4 formalization with other theorem proving environments such as Isabelle/HOL, we identified significant differences in the design decisions that impact efficiency, maintainability, and complexity. While Isabelle opts for an explicit height parameter stored within each node, HOL4 dynamically computes the height during recursive operations. This leads to a trade-off between the simplicity of the AVL tree structure and the performance of height lookups during rebalancing operations. The comparative analysis highlights both the strengths and limitations of each approach, providing insights into different formalization strategies.

This formalization is not only an academic exercise but also a practical contribution to the ongoing development of verified data structures in HOL4. The methods and techniques developed here can be extended to other self-balancing tree structures or more complex data structures, further enriching the HOL4 ecosystem. Future work can build on this foundation, integrating AVL trees into more advanced applications such as formalized compilers, automated reasoning tools, or verified databases.

In conclusion, this thesis advances the state of the art in formal verification by providing a rigorous, reusable, and modular formalization of AVL trees in HOL4. It serves as a stepping stone for future developments, encouraging further exploration and refinement of formally verified data structures within theorem proving systems.